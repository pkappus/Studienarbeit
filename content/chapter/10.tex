\chapter{Ausblick}
\label{chap:ausblick}
Trotz der sehr aussagekräftigen Ergebnisse gibt es viel Raum für Optimierung. Um diese Arbeit auf noch festere Füße stellen zu können, wäre eine statistische Verifikation der Hypothesen, welche in Kapitel \ref{sec:hypothesen} vorgestellt wurden. Anhand eines händisch erstellten Datensatzes könnte so herausgefunden werden, in wie viel Prozent ein Retweet tatsächlich einer Zustimmung entspricht und in wie viel Prozent das Gegenteil der Fall ist. Diese Erkenntnisse könnten so unsre Berechnungen mehr an die Realität annähern. \\ \newline
Ein wichtiger Punkt ist die Verbesserung der Sentimentanalyse. Die Stimmung des Verfassers aus einem Text abzuleiten ist um ein einiges komplexer als das Vorkommen eine Wortes zu überprüfen, weshalb deutlich komplexere und adaptiver Algorithmik benötigt wird um Sinnvolle Ergebnisse zu erzielen. Die jüngsten Vorschritte im Bereich der Verarbeitung natürlicher Sprache zeigen aber, dass dies mit aktuellem Stand der Technik durchaus realisierbar sein sollte. Durch die Verwendung von modernen Modellen der Sprachverarbeitung wie zum Beispiel der Watson KI \cite{tone_ibm}, welche auch über eine Stimmungsanalysefunktion verfügt könnten Stimmungen deutlich differenzierter und realitätsnäher herausfinden. Allerdings ist die Verwendung dieser kostenpflichtig, eine Analyse des hier verwendeten Datensatzes wäre im Kontext dieser Studienarbeit zu teuer. Ein weiteres Problem ist, dass viele der kostenlosen Lösungen zwar im Englischen sehr gute Ergebnisse liefern, für die Deutsch aber nur schlecht anwendbar sind. Eine Möglichkeit für die Verbesserungen der Methodik zur Sentimentanalyse allgemein wäre, die Berechnung der Stimmung des Nutzers durch den die durchschnittliche Stimmung seiner Tweets um eine detailliertere Methode zu ersetzten: Sinnvoller könnte es sein, die Stimmung eines Tweets in Verbindung mit in ihm verwendeten Schlüsselwörter zu verbinden um so herauszufinden, in welchem Verhältnis der Nutzer zu einem bestimmten Wort steht, statt nur auf eine allgemeines Verhältnis zum Thema. Letztendlich finden wir alle Corona nervig, der eigentliche Erkenntnisgewinn könnte sich eine Ebene darunter verbergen. Diesem könnte in weiteren Untersuchungen nachgegangen werden. \\ \newline
Auch für bei den anderen Vorgehensweisen gibt es Raum für Verbesserungen: Das Auffinden von Clustern im Retweet-Netzwerk hängt stark von der Anzahl der Influencer und dem gewählten Schwellenwert ab. In zukünftigen Arbeiten eine Methode zur Wahl dieser Parameter, um die Komplexität nur so weit wie nötig zu reduzieren und dabei möglichst wie nötig zu reduzieren und dabei möglichst viele Nutzer im Datensatz zu behalten, wird einen Schritt nach vorne bedeuten. Die Ausweitung der Sprachcluster-Methode auf Schlüsselwörter und Feinabstimmung der Parameter für k-means und DB-SCAN kann mehr eindeutige Cluster ergeben. Ein wichtiger Faktor ist die Anzahl der Iterationen des k-Means-Algorithmus. Die fünfzehn Iterationen in diesem Fall dauerte die Berechnung etwa 23 Stunden, aber es gibt gibt es Möglichkeiten für Optimierungen. Mehr Iterationen mit Werten für k $\geq$ 20 oder mehr würden die Ergebnisse noch detaillierter machen. Eine Vergrößerung des gesamten Datensatzes auf mehr als einen Monat würde bessere und deutlichere Ergebnisse liefern. Außerdem, da beide Methoden einfache Ordner- und Dateioperationen verwenden, könnte eine Architektur aufgebaut werden, um die Rechenzeit zu verringern. Zeit. Viele Aufgaben sind auch theoretisch parallelisierbar, in die Praxis umzusetzen, was die Rechenzeit ebenfalls verringern würde. \\ \newline
Ein wichtiger Schritt wäre die Generalisierung des nun dargestellten Prozesses: Satt einem Datensatz aus Tweets zum Thema Corona zu verwenden könnten diese auch einem ganz anderen Themengebiet entstammen. So könnte man zum Beispiel einen Datensatz zum Thema Bundestagswahl dazu verwenden, Wähler einzelner Parteien zu identifizieren. Auch eine Marketingabteilung könnte die Wahrnehmung der eigenen Marke in der Twittercommunity besser nachvollziehen. Theoretisch sollte der Prozess auf jedes Thema anwendbar sein, jedoch wurde dies noch nicht in der Praxis überprüft. \\ \newline
Wir hoffen, in dieser Arbeit einen Beitrag zur Analyse der Struktur des Twitter-Ökosystems erarbeitet zu haben und hoffen, dass weitere Forschung aufbauend auf unserer Arbeit durchgeführt werden kann. Wie nun dargestellt, ist weder das Thema, noch der Datensatz und auch unsere Ideen noch lange nicht ausgeschöpft. Wir hoffen, uns auch in der Zukunft noch weiter mit dem Thema beschäftigen zu könne und bedanken uns herzlichst bei der Dualen Hochschule Baden-Württemberg und unserem Betreuer für die Unterstützung. 

