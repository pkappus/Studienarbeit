\chapter{Clustering Retweets}
\label{chap:cluster_retweets}
Wie in Abschnitt \ref{sec:struktur-eines-tweets} beschrieben enthält ein \gls{Retweet} die Informationen des/der Nutzer*in der/die ihn veröffentlicht hat, sowie die Informationen des/der Nutzer*in der/die den Original-Tweet veröffentlicht hat. Eine Beziehung kann hergestellt werden. Außerdem kann durch die Anzahl an \glspl{Retweet} der Beziehung eine Gewichtung zugeschrieben werden. Durch Analyse aller \glspl{Retweet} kann so ein Netz von Beziehungen zwischen Nutzern aufgebaut und anschließend auf enthaltenen Gruppierungen untersucht werden. Im Folgenden soll diese Methode dargestellt und die Ergebnisse präsentiert werden.

\section{Daten vorbereiten}
\label{sec:daten-vorbereiten}

