\chapter{Rückblick}
\label{chap:ruckblick}
Twitter ist eine Plattform, auf der Millionen von Menschen jeden Tag öffentlich ihre Gefühle und Meinungen teilen. Die Analyse selbst großer Mengen dieser Informationen ist durch die Zunahme der Computerleistung möglich geworden, aber auch durch die Hilfe vieler engagierter Forscher, die immer bessere Wege finden, diese ansonsten chaotischen Daten zu strukturieren. \\ \newline
In diesem Beitrag wurden zwei Ansätze für das Problem des Auffindens von Clustern in diesem unstrukturierten Datensatz vorgestellt. Wir konnten zeigen, dass es Cluster auf Twitter gibt, die nur sich selbst retweeten und dieselbe Sprache bezüglich der Debatte über die Covid-19-Pandemie verwenden. \\ \newline
Mit Hilfe der Sentimentanalyse ist es uns aus den in Kapiteln \ref{chap:sentiment}  genannten Gründen schwer gefallen aussagekräftige Ergebnisse zu errechnen. Eine sinnvolle Einteilung in Gruppen gelang nicht. Es ist jedoch noch viel Raum für Verbesserung, auf den wir in folgenden Kapitel noch weiter eingehen wollen.

