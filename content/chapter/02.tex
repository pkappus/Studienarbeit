\chapter{Vorüberlegungen}
\label{chap:voruberlegungen}
In diesem Kapitel sollen zunächst die Hypothesen dargestellt werden, anhand  derer die in der Fragestellung formulierten Fragen potentiell beantworten kann. Von diesen wird dann eine Vorgehensweise abgeleitet.
\section{Hypothesen}
\label{sec:hypothesen}
Will man Objekte in Gruppen einteilen, so muss man zunächst definieren, anhand welcher Merkmale man dies umsetzen will. Bei einer Menge an Punkten aus dem zweidimensionalen Raum bedient man sich einfach deren Ortsvektor. Bei einer Menge an Twitter Nutzer*innen  ist dies schwieriger. Diese Arbeit versucht, diese anhand von drei Merkmalskategorien einzuteilen, die nun hypothetisch erläutert werden: \\ \newline
Zum einen wird davon ausgegangen, dass Nutzer*innen sich anhand ihrer positiven bzw. negativen Einstellung zu einem bestimmten Thema kategorisieren lassen können sollten. Durch eine Sentimentanalyse aller Tweets eines*r Nutzers*in kann auf die Stimmung der Person geschlossen werden. Vergleicht man die Stimmungsbilder der Nutzer*innen untereinander, so könnten sich Kategorien ergeben. Ist dies Realisierbar, könnte auch über den zeitlichen Verlauf die Stimmung in Deutschland zum Thema Corona dargestellt werden. In Kapitel \ref{chap:sentiment} wird dieser Hypothese nachgegangen. \\ \newline
Ein weiteres markantes Merkmal könnte sein, wen ein bestimmte*r Nutzer*in wie oft retweetet, also den Text eines*r anderen*en  (manchmal auch kommentiert) an seine/ihre eigene Followerschaft weiterleitet. Retweetet eine Person eine andere, so ist sie der Meinung, dass der Tweet der/des Anderen es wert ist, (kommentiert) weiter propagiert zu werden. Dies könnte einer Zustimmung gleichgesetzt werden, jedoch kann es gut sein, dass das eigne Kommentar die Aussage des original Tweets kritisiert. In diesem Fall ist das Gegenteil der Fall. In der Hoffnung, dass dieser Effekt statistisch gesehen keine großen Auswirkungen hat wird diese Theorie in \ref{chap:cluster_retweets} in die Praxis umgesetzt. \\ \newline
Die dritte Merkmalskategorie betrifft die Sprache, präziser die Worte die verwendet werden. Die Sprache einer Person, also welche Worte sie in welchem Zusammenhang wann wählt, ist eine wichtige Charaktereigenschaft. Zum einen dient sie häufig als Merkmal der gesellschaftlichen Klassenzugehörigkeit, zum anderen definieren die Wörter auch die Themen mit denen wir uns auseinandersetzen. Man kann nicht über den Klimawandel diskutieren, ohne sich des Wortes \textit{Klimawandel} zu bedienen. Somit sollten alle Personen, bei welche das Wort Klimawandel in ihren Tweets zu finden ist, sich auch mit dem Klimawandel beschäftigen. Man kann sie also anhand der Verwendung oder Nichtverwendung dieses Wortes klassifizieren. Eine Einschränkung ist jedoch, dass man dadurch zwar bestimmen kann, ob sich jemand mit einem bestimmten Thema beschäftigt, aber nicht, welche Meinung er/sie zu diesem hat. Dies könnte sich aber aus der Summen seiner/ihrer Themen ergeben. Diesem Ansatz soll in Kapitel \ref{chap:cluster_hashtags} nachgegangen werden. 
\section{Vorgehensweise}
\label{sec:vorgehensweise}
Ein detailliertes Vorgehen ist aufgrund der hohen Komplexität des Sachverhalts schwer zu definieren und noch schwerer einzuhalten. Grundsätzlich muss zunächst der Datensatz agglomeriert werden. Anschließend soll nach einem agilen Ansatz jeder der drei Hypothesen nachgegangen werden. Dabei soll eine Arbeitsaufteilung nach Hypothese erfolgen. Des weiteren soll zunächst nach dem Prinzip des Most Valuable Products ein Walking Skeleton der Analysemethoden erschaffen werden, bevor man sich um eine detaillierte Optimierung der Algorithmen bemüht. Somit soll verhindert werden, dass in etwas viel Arbeit gesteckt wird, welches sich im ganzen nicht realisieren lässt. Alle wichtigen Erfolge und Fehlschläge sollen mit dem Projektteam kommuniziert werden.