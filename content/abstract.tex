%!TEX root = ../main.tex

\pagestyle{empty}

% override abstract headline
\renewcommand{\abstractname}{Abstract}

\begin{abstract}
In this paper we investigate the use of Open-Source sentimentanalysis tools in terms of clustering users on Twitter. Furthermore we present two clustering methods to determine users with similar opinions and common topics on the Covid-19 pandemic and the related public debate in Germany. We believe, they can help gaining an overview over similar-minded groups and could support the prevention of fake-news distribution.
The first method uses a new approach to create a network based on retweet-relationships between users and the most retweeted characters. The second method extracts hashtags from users posts to create a user feature vector which is then clustered using a combination of the k-Means and DB-SCAN algorithm  to identify groups using the same language. With both approaches it was possible to identify clusters that seem to fit groups of different public opinion in Germany. However, we also found that clusters from one approach can not be associated with clusters from the other due to the filtering steps in the two methods.
\end{abstract}