%!TEX root = ../main.tex

%
% To create glossary run the following command: 
% makeglossaries main.acn && makeglossaries main.glo
%

%
% Glossareintraege --> referenz, name, beschreibung
% Aufruf mit \gls{...}
%
\newglossaryentry{Glossareintrag}{name={Glossareintrag},plural={Glossareinträge},description={Ein Glossar beschreibt verschiedenste Dinge in kurzen Worten}}

\newglossaryentry{Amazon Web Services}{name={Amazon Web Services},plural={Amazon Web Services},description={Cloud-Infrastruktur die Server, Speicher uvm. einfach und kostengünstig bereitstellt }}
\newglossaryentry{Application Programming Interface}{name={Application Programming Interface},plural={Application Programming Interfaces},description={Von Softwaresystemen zur Verfügung gestellte Schnittstelle für andere Programme um zu interaktieren}}
\newglossaryentry{EC2 - Instanz}{name={EC2 - Instanz},plural={EC2 - Instanzen},description={Ein virtueller Server, der bei AWS gemietet werden kann}}
\newglossaryentry{S3 - Bucket}{name={S3 - Bucket},plural={S3 - Buckets},description={Ein Filehosting Service auf dem über HTTPS Daten in Ordnerstrukturen abgelegt und heruntergeladen werden können}}

\newglossaryentry{Amazon Kinesis Data Firehose}{name={Amazon Kinesis Data Firehose},plural={Amazon Kinesis Data Firehose},description={Ein Service von AWS, der Streaming Daten aufnehmen, transformieren und in Datenspeicher ablegen kann}}
\newglossaryentry{Hashtag}{name={Hashtag},plural={Hashtags},description={Ein Schlagwort angeführt von einem Doppelkreuz(\#), das dazu dient Texte mit einem bestimmten Thema zu versehen}}	
\newglossaryentry{JavaScript Object Notation}{name={JavaScript Object Notation},plural={JavaScript Object Notation},description={Ein Datenformat aufgebaut in hierarischer Form das Zeichenketten, Zahlen, Listen und weitere Objekte erlaubt}}	
\newglossaryentry{Retweet}{name={Retweet},plural={Retweets},description={Ein veröffentlicher Tweet kann von anderen Nutzer auf deren Seite veröffentlicht werden. Dieser neue Tweet, der dem Original gleich ist, wird Retweet genannt}}	
\newglossaryentry{Text Mining}{name={Text Mining},plural={Text Mining},description={Ein Feld der Informatik das sich mit der Aufgabe beschäftig, mit Computern Informationen und Wissen aus Texten zu erhalten}}	
\newglossaryentry{Natural Language Processing}{name={Natural Language Processing},plural={Natural Language Processing},description={Die Methodiken mit denen, Computer Texte analysieren}}	
\newglossaryentry{OpenSource}{name={OpenSource},plural={OpenSource},description={Software deren Quellcode öffentlich zugänglich ist, meistens auch konstenlos}}	
%\newglossaryentry{}{name={},plural={},description={}}
%\newglossaryentry{}{name={},plural={},description={}}
%\newglossaryentry{}{name={},plural={},description={}}
%\newglossaryentry{}{name={},plural={},description={}}
%\newglossaryentry{}{name={},plural={},description={}}
%\newglossaryentry{}{name={},plural={},description={}}
%\newglossaryentry{}{name={},plural={},description={}}
%\newglossaryentry{}{name={},plural={},description={}}
%\newglossaryentry{}{name={},plural={},description={}}
%\newglossaryentry{}{name={},plural={},description={}}